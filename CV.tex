\documentclass{resume} % Use the custom resume.cls style

\usepackage[left=0.5in,top=0.5in,right=0.5in,bottom=0.5in]{geometry} % Document margins
\newcommand{\tab}[1]{\hspace{.2667\textwidth}\rlap{#1}} 
\newcommand{\itab}[1]{\hspace{0em}\rlap{#1}}
\name{OMAR BARBIERO} % Your name
%\address{123 Pleasant Lane \\ City, State 12345} % Your secondary addess (optional)
\address{Federal Reserve Bank of Boston \\ omar.barbiero@bos.frb.org}  % Your phone number and email

\begin{document}

%----------------------------------------------------------------------------------------
%	OBJECTIVE
%----------------------------------------------------------------------------------------

\begin{rSection}{EMPLOYMENT}

{\bf Federal Reserve Bank of Boston} \hfill {2020 - \textit{present}}\\
Economist

\end{rSection}
%----------------------------------------------------------------------------------------
%	EDUCATION SECTION
%----------------------------------------------------------------------------------------

\begin{rSection}{Education}

{\bf Harvard University} \hfill {2020}
\\ 
PhD in Economics
\\
\hspace*{0.3cm} \textit{Fields}: International Macroeconomics, Macroeconomics, Trade


{\textbf{Bocconi University}}  \hfill 2013\\
MSc in Economics
 
{\textbf{University of Padua}}  \hfill 2010\\
BSc in Statistics

%Minor in Linguistics \smallskip \\
%Member of Eta Kappa Nu \\
%Member of Upsilon Pi Epsilon \\


\end{rSection}


%----------------------------------------------------------------------------------------
%	WORKING PAPERS
%----------------------------------------------------------------------------------------

\begin{rSection}{WORKING PAPERS}

{\textbf{The Valuation Effects of Trade}\hfill{2020}}\\\textit{Job Market Paper}\\
{This paper estimates the cash flow effects of currency mismatches generated by foreign-priced operations of French manufacturers. The value of transactions invoiced in foreign currencies is twice as sensitive to exchange rates as the value of transactions invoiced in the domestic currency. I aggregate foreign-priced operations to the firm level to build a shift-share measure of invoice currency mismatch. This measure outperforms any trade-weighted effective exchange rate index at explaining cash flows of trading firms. However, virtually all investment and payroll sensitivity to exchange rates due to measured invoice currency mismatch come from small domestic-oriented firms. The real macroeconomic effects are limited because large traders are liquid and small exporters partially hedge their dollar-priced exports with dollar-priced imports. These results show how large trade value sensitivities to currency fluctuations can coexist with the evidence of disconnect between exchange rates and real macroeconomic fundamentals.}


\bigskip

\textbf{The Effects of Fiscal Consolidations: Theory and Evidence}\hfill2018\\
\textit{with A. Alesina, C. Favero, F. Giavazzi, M. Paradisi}\\
{We investigate the macroeconomic effects of fiscal consolidations based upon government spending cuts, transfers cuts and tax hikes. We extend a narrative dataset of fiscal consolidations, with details on over 3500 measures for 16 OECD countries. We show that government spending cuts and cuts in transfers are 	much less harmful than tax hikes, despite the fact that non-distortionary transfers are not classified as spending. Standard New Keynesian models robustly match our results when fiscal shocks are persistent. 	Wealth effects on aggregate demand mitigate the impact of a persistent spending cut. Static distortions caused by persistent tax hikes cause larger shifts in aggregate supply under sticky prices.}

\end{rSection} 



%----------------------------------------------------------------------------------------
%	PUBLICATIONS
%----------------------------------------------------------------------------------------

\begin{rSection}{PUBLICATIONS}
	
	{\textbf{The Macroeconomics of Border Taxes}\hfill{2018}}\\\textit{with E. Farhi, G. Gopinath, and O. Itskhoki}. NBER Macroeconomics Annual. 33: 395-457\\

	
	\textbf{Austerity in 2009-2013}\hfill 2015\\
	\textit{with A. Alesina, C. Favero, F. Giavazzi, M. Paradisi.} Economic Policy, 30, 83: 383-437\\

	
\end{rSection} 






\begin{rSection}{PROFESSIONAL ACTIVITIES} 
\begin{rSubsection}{Presentations}{}{}{}
	\item Online International Finance and Macro Seminar \hfill 2020
	\item EIEF, Rome \hfill 2019
	\item NBER Macroeconomic Annual \hfill 2018
\end{rSubsection}

\begin{rSubsection}{Referee Service}{}{}{}
	\item Journal of International Economics
	\item Quarterly Journal of Economics
\end{rSubsection}

\end{rSection} 

\begin{rSection}{TEACHING} 
\textbf{International Finance}, Graduate, Harvard \hfill 2017-2019 \\
Teaching Fellow for Professors Gopinath, and Maggiori

\textbf{A Libertarian Perspective on Economic and Social Policy}, Undergraduate, Harvard \hfill 2017-2018 \\ Teaching Fellow for Professor Miron

\textbf{The Future of Globalization}, Undergraduate, Harvard \hfill 2017 \\
Teaching fellow for Professors Summers and Lawrence
\end{rSection}


\bigskip

\begin{rSection}{HONORS, SCHOLARSHIPS \& FELLOWSHIPS} 

Molly and Domenic Ferrante Economics Research Fund, Harvard University \hfill 2019 \\
Certificate of Distinction in Teaching, Harvard University \hfill 2018-19 \\
Research Grant, Harvard Institute for Quantitative Social Science \hfill 2018-19 \\
Jens Aubrey Westengard Fund, Harvard University \hfill 2018 \\
Research Grant, Weatherhead Center Mid-dissertation Grant, Harvard University \hfill 2018 \\
Research Grant, Lab for Economic Applications and Policy, Harvard University \hfill 2017 \\

\end{rSection}


\begin{rSection}{RESEARCH EXPERIENCE \& PRE-DOC EMPLOYMENT}
	\textbf{Research Assistant}\\
	Harvard University, for Professor Gopinath \hfill 2016 \\
	Bocconi University, for Professors Alesina, Favero, and Giavazzi \hfill 2013-2014 \\
	
	\textbf{Consultant}\\
	Organization for Economic Co-operation and Development (OECD) \hfill 2012-2014 \\
	\renewcommand\labelitemi{$\cdot$}
	\begin{itemize}[leftmargin=*]
		\vspace{-1.5em}
		\setlength\itemsep{-.5em}
		\item \textbf{The 2013 update of the OECD's database on product market regulation} \hfill 2015 \\ \textit{with I. Koske, I. Wanner, R. Bitetti}. OECD Economics Working Papers No. 1200
		\item \textbf{New econometric estimates of long-term growth effects of different areas of public spending} \hfill 2013 \\ \textit{with B Cournède}. OECD Economics Working Papers No. 1100
		\item \textbf{Boosting productivity in Australia} \hfill 2013 \\ \textit{with V. Koutsogeorgopoulou}. OECD Economics Working Papers No. 1025
	\end{itemize}
	
\end{rSection}


\end{document}
